\documentclass{article}
\usepackage{amsmath}
\usepackage{amsthm}
\usepackage{amssymb}

\makeatletter
\renewcommand{\abstractname}{Instructions}
\makeatother

\newtheorem*{theorem}{Theorem}

\title{MAT/CSC -- 220: Discrete Structures\\
\large{EFY}}
\author{Thomas R. Cameron}
\date{March 23, 2018}

\begin{document}
\maketitle

\paragraph*{Problem 1.} Use structural induction to verify the correctness of the following function:
\begin{align*}
&val~rec~height\colon~tree\rightarrow~int~=\\
&fn~Leaf~n~=>~0 \\
&|~Internal(n,x,y)~=>~1+Int.max(height(x),height(y))
\end{align*}
Note that max is a member function of the integer signature of SML. 

\paragraph*{Problem 2.} Use structural induction to prove the following result.

\begin{theorem}
For any full binary tree $T$, the number of nodes in $T$ is odd. 
\end{theorem}

\paragraph*{Discussion Points.} Below is a list of interesting topics that relate to trees and structural induction. 
\begin{itemize}
\item	Structural Induction makes it easy to verify the correctness of operations on recursively defined data structures. Therefore, making recursive functions elegant solutions to problems. 
\item	Recursion is done ``properly'' in a functional programming paradigm. That is, by treating computation as the evaluation of mathematical functions and avoiding changing state and mutable data we are able to run recursive functions without hitting a recursion depth limit and avoiding stack overflow. 
\item	In order to fully optimize recursion, any two term recurrence relations should be modified to a one term relation. This is what is meant by tail recursion. 

For instance, consider the example below.
\begin{align*}
&val~rec~fib1~:~int\rightarrow int~=\\
&\hspace*{2em} fn~0~=>~1\\
&\hspace*{2em} |~1~=>~1\\
&\hspace*{2em} |~n:int~=>~fib1(n-1)+fib1(n-2)
\end{align*}
Fib1 is a recursive function that evaluates Fibonacci numbers in perhaps the most natural way. However, this natural definition has an exponential cost complexity. Whereas, the Fib2 implementation of the Fibonacci numbers has a linear cost complexity. 
\begin{align*}
&local\\
&\hspace*{2em}val~rec~helper~:~int\rightarrow int*int~=\\
&\hspace*{2em}fn~0~=>~(1,0)\\
&\hspace*{2em}|~1~=>~(1,1)\\
&\hspace*{2em}|~n:int~=>\\
&\hspace*{4em}let\\
&\hspace*{6em}val~(a:int,b:int)~=~helper(n-1)\\
&\hspace*{4em}in\\
&\hspace*{6em}(a+b,a)\\
&\hspace*{4em}end\\
&in\\
&\hspace*{2em}val~fib2~:~int\rightarrow int~=\\
&\hspace*{2em}fn~n:int~=>\\
&\hspace*{4em}let\\
&\hspace*{6em}val (a:int,\_) = helper(n)\\
&\hspace*{4em}in\\
&\hspace*{6em}a\\
&\hspace*{4em}end\\
&end\\
\end{align*}
\end{itemize}


 
\end{document}
