\documentclass{article}
\usepackage{amsmath}
\usepackage{amsthm}
\usepackage{amssymb}
\usepackage{enumerate}

\makeatletter
\renewcommand{\abstractname}{Instructions}
\makeatother

\title{MAT -- 112: Calculus I and Modeling\\
\large{Logarithm and Exponent Derivatives}}
\author{Instructor: Thomas R. Cameron}
\date{February 21, 2018}

\begin{document}
\maketitle

\begin{abstract}
Read the section on the natural exponent derivative. Use the writing style I use in the proof as a guideline for the next two sections, where you are asked to
prove derivative rules for exponent and logarithm functions.
\end{abstract}

\paragraph*{Natural Exponent Derivative.} The following holds true
\[
\frac{d}{dx}e^{x}=e^{x}.
\] 
\begin{proof}
Using the limit definition of the derivative we have
\begin{align*}
\frac{d}{dx}e^{x}&=\lim_{h\rightarrow 0}\frac{e^{x+h}-e^{x}}{h} \\
&=\lim_{h\rightarrow 0}\frac{e^{x}\left(e^{h}-1\right)}{h} \\
&=e^{x}\lim_{h\rightarrow 0}\frac{e^{h}-1}{h}.
\end{align*}
All that remains is to show that $\lim_{h\rightarrow 0}\frac{e^{h}-1}{h}=1$. To this end, we introduce the change of variables
\[
u=e^{h}-1,
\]
which implies that $\ln(u+1)=h$. Since $u\rightarrow 0$ as $h\rightarrow 0$ we have
\[
\lim_{h\rightarrow 0}\frac{e^{h}-1}{h}=\lim_{u\rightarrow 0}\frac{u}{\ln(u+1)}. 
\]
Next, we multiply the numerator and denominator by $(1/u)$ to give
\[
\lim_{u\rightarrow 0}\frac{1}{\frac{1}{u}\ln(u+1)} = \lim_{u\rightarrow 0}\frac{1}{\ln(1+u)^{1/u}}.
\]
Let $n=1/u$ and note that $n\rightarrow\infty$ as $u\rightarrow 0$. Therefore, we have
\[
\lim_{n\rightarrow\infty}\frac{1}{\ln(1+\frac{1}{n})^{n}}=\frac{1}{\ln(e)}=1.
\]
\end{proof}

\paragraph*{Exponent Derivative.}	Use the previous result to show that
\[
\frac{d}{dx}a^{x}=(\ln a)a^{x}.
\]

\vspace{17em}

\paragraph*{Logarithm Derivative.}	Use the previous result to show that
\[
\frac{d}{dx}\log_{a}(x) = \frac{1}{(\ln a) x}.
\]
Then deduce formula for $\frac{d}{dx}\ln(x)$. 

\newpage

\paragraph*{Applications.}
\begin{enumerate}
\item	The altitude (feet) of a plane $t$ (min) after takeoff is given by
\[
h(t)=2000\ln(t+1),~\quad~0\leq t\leq 5.
\]

\vspace*{10em}

\item	The sound pressure $P$ (dB) for a given sound can be modeled by
\[
P=20\log_{10}\frac{W}{W_{0}},
\]
where $W$ is the size of the variable energy source and $W_{0}$ is a constant. 
Compute $\frac{dP}{dt}$ at $t=3$, if $W=7.2$ and $\frac{dW}{dt}=0.5$ at $t=3$. 

\vspace*{10em}

\item	The charge of a capacitor in a circuit containing a capacitance $C$, resistance $R$, and source voltage $E$ is given by
\[
q=CE\left(1-e^{-t/RC}\right).
\]
Show that the follow equation holds true
\[
R\frac{dq}{dt}+\frac{q}{C}=E
\]
\end{enumerate}

\end{document}