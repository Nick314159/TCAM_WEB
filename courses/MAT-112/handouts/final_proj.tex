\documentclass{article}
\usepackage{amsmath}
\usepackage{amsthm}
\usepackage{amssymb}
\usepackage{enumerate}

\makeatletter
\renewcommand{\abstractname}{Instructions}
\makeatother

\title{MAT -- 112: Calculus I and Modeling\\
\large{Final Project}}
\author{Instructor: Thomas R. Cameron}
\date{April 6, 2018}

\begin{document}
\maketitle

\begin{abstract}
 This handout is intended to prepare you for the upcoming final project. 
\end{abstract}

\paragraph*{Final Project Outline.} The final project will be graded as follows:
\begin{itemize}
\item	[(10 pts.)] You must select a group of three to at most four people, a topic, and date you will present. Inform me of your choice via email by Wednesday, April 18; your choice must include a clever and short presentation title. You will be deducted 1 pt for each day past 4/18 for which you don't inform me of your choices. Keep in mind the dates to present are first come first serve.

\item	[(15 pts.)] Your research must include at least 3 sources, one of which must be a book outside of your textbook.
\item [(20 pts.)] The day your present you must turn in a  one page outline of your talk that includes major findings from your research.
\item	[(25 pts.)] You will be graded on your group participation. I will review your participation during the presentation and each group member will fill out a short rubric on their fellow group members participation throughout the entire project. As a part of your participation you are expected to attend each final presentation (barring excused absences) and you are expected to be respectful while giving your complete attention to the presenters. In addition, you are expected to ask at least one question throughout the course of all final presentations.
\item	[(30 pts.)] You will be graded on your group presentation as outlined in the rubric below. 
\end{itemize}
\newpage
\paragraph*{Presentation Rubric.} During your presentation you will be graded on the following:
\begin{itemize}
\item	(5 pts.) Information is presented in a clear and logical form.
\item	(5 pts.) Introduction is attention-getting, lays out motivation for the selected topic, and establishes framework for the rest of the presentation.
\item (10 pts.) Presentation contains accurate information that is relevant to the topic and is explained in a succinct and precise manner.
\item	(5 pts.) Each person in the group speaks while maintaining good eye contact with the audience and being appropriately animated. 
\item	(5 pts.) Length of presentation is within the assigned time limits (15 min.), no shorter and no longer while leaving room for questions. 
\end{itemize}

\paragraph*{Topic Requirements.} You must pick a topic that relates to our course and hopefully connects our course material with something else you are interested in and passionate about. Potential topics are listed below.
\begin{itemize}
\item	Any of the material from Sections 9.1 -- 9.5, 13.1 -- 13.3, and 14.1 -- 14.3.
\item	Any applications of the material we covered to the life or physical sciences (e.g. see the Extended Applications at the end of each chapter).
\item	Any other topic of your choosing, with my approval. 
\end{itemize}
\end{document}