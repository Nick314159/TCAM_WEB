\documentclass{article}
\usepackage{amsmath}
\usepackage{amsthm}
\usepackage{amssymb}
\usepackage{enumerate}
\usepackage{pgfplots}

\makeatletter
\renewcommand{\abstractname}{Instructions}
\makeatother

\pgfplotsset{holdot/.style={color=blue,fill=white,only marks,mark=*}}

\title{MAT -- 112: Calculus I and Modeling\\
\large{Numerical Integration}}
\author{Instructor: Thomas R. Cameron}
\date{April 20, 2018}

\begin{document}
\maketitle

\begin{abstract}
Below is a review of the numerical integration techniques you should know from this course. For each rule, the interval $[a,b]$ is being split into $n$ subintervals of length $h=(b-a)/n$. The subintervals are denoted by $[x_{i},x_{i+1}]$ for $i=0,1,2,\ldots,n-1$, where $a=x_{0}<x_{1}<x_{2}<\cdots<x_{n}=b$. 
\end{abstract}

\paragraph*{Left-Hand Rule.} The definite integral is approximated as follows:
\[
\int_{a}^{b}f(x)dx\approx h\left(f(x_{0})+f(x_{1})+\cdots+f(x_{n-1})\right).
\]

\paragraph*{Trapezoidal Rule.} The definite integral is approximated as follows:
\[
\int_{a}^{b}f(x)dx\approx h\left(\frac{1}{2}f(x_{0})+f(x_{1})+\cdots+f(x_{n-1})+\frac{1}{2}f(x_{n})\right)
\]

\paragraph*{Simpson's Rule.} The definite integral is approximated as follows:
\begin{align*}
\int_{a}^{b}f(x)dx\approx \frac{h}{6}\biggl(f(x_{0})+4f\left(\frac{x_{0}+x_{1}}{2}\right)+2f(x_{1})&+4f\left(\frac{x_{1}+x_{2}}{2}\right)+\cdots \\
&+2f(x_{n-1})+4f\left(\frac{x_{n-1}+x_{n}}{2}\right)+f(x_{n})\biggr)
\end{align*}

\end{document}