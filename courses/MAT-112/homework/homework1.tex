\documentclass{article}
\usepackage{enumerate}
\usepackage{siunitx}

\makeatletter
\renewcommand{\abstractname}{Instructions}
\makeatother

\title{MAT -- 112: Calculus I and Modeling\\
\large{Homework 1}}
\author{Instructor: Thomas R. Cameron}
\date{Due: 1/26/2018}

\begin{document}
\maketitle

\begin{abstract}
You must complete all book problems and other problems. The book problems are intended to give you practice in solving problems from the textbook. They are graded based upon completion and correctness. The other problems are intended to help further your understanding of the concepts from a theoretical point of view. These problems are more rigorously graded, with a high expectation on the student providing clear, detailed, and justified answers. Lastly, you may work with other students and ask me any questions, but you may not look up solutions online. You must write your solutions independently so I may interpret your understanding while grading. 
\end{abstract}

\subsection*{Book Problems}
\begin{itemize}
\item   [\S 1.1:] 29, 32, 36, 78
\item   [\S 1.3:] 29, 37
\item   [\S 1.4:] 7, 35, 49
\item   [\S 1.5:] 49
\end{itemize}

\subsection*{Other Problems}

\paragraph*{Problem 1.}	Use the definition of perpendicular lines from the book in order to show that two lines are perpendicular if and only if they intersect at a \ang{90} angle.\\
\emph{Your explanation should include a clearly labeled diagram}. 

\paragraph*{Problem 2.}	Use the definition of a function from the book in order to show that if $f$ is a function from $A$ onto all of $B$ and $g$ is a function from $B$ to $C$, then $h=g\circ f$ is a function from $A$ to $C$.

\paragraph*{Problem 3.}	Use the method of completing the square to transfer the quadratic $ax^{2}+bx+c$ from standard form to vertex form. Once in vertex form, identify the vertex, axis of symmetry, $x$-intercept, and $y$-intercept.



\end{document}
