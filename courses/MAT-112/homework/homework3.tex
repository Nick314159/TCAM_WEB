\documentclass{article}
\usepackage{enumerate}
\usepackage{siunitx}

\makeatletter
\renewcommand{\abstractname}{Instructions}
\makeatother

\title{MAT -- 112: Calculus I and Modeling\\
\large{Homework 3}}
\author{Instructor: Thomas R. Cameron}
\date{Due: 2/16/2018}

\begin{document}
\maketitle

\begin{abstract}
You must complete all book problems and other problems. The book problems are intended to give you practice in solving problems from the textbook. They are graded based upon completion and correctness. The other problems are intended to help further your understanding of the concepts from a theoretical point of view. These problems are more rigorously graded, with a high expectation on the student providing clear, detailed, and justified answers. Lastly, you may work with other students and ask me any questions, but you may not look up solutions online. You must write your solutions independently so I may interpret your understanding while grading. 
\end{abstract}

\subsection*{Book Problems}
\begin{itemize}
\item   [\S 3.1:] 2, 6, 10, 36. 38, 44, 91
\item   [\S 3.2:] 2, 6, 10, 24, 28, 30, 35
\item   [\S 3.3:] 4, 6, 10, 17, 20, 25, 31 \\
\emph{For 4--10 use the definition on p. 156, for 17--20 evaluate the limit by hand and use the definition on p. 159, for 25 use $h=0.001$ and $h=-0.001$ and average the two results, and for 31 (b) evaluate the limit by hand.}
\item   [\S 3.4:] 6, 7, 10, 22, 26, 39, 51 (a)--(b)
\end{itemize}

\subsection*{Other Problems}

\paragraph*{Problem 1.} Use the definition of the derivative on p. 172 to derive formula for the derivative of $x^{2}$, $\sqrt{x}$, and $1/x$. Then use this formula to determine the domain of each derivative function. 

\end{document}