\documentclass{article}
\usepackage{enumerate}
\usepackage{siunitx}

\makeatletter
\renewcommand{\abstractname}{Instructions}
\makeatother

\title{MAT -- 112: Calculus I and Modeling\\
\large{Homework 4}}
\author{Instructor: Thomas R. Cameron}
\date{Due: 2/23/2018}

\begin{document}
\maketitle

\begin{abstract}
You must complete all book problems and other problems. The book problems are intended to give you practice in solving problems from the textbook. They are graded based upon completion and correctness. The other problems are intended to help further your understanding of the concepts from a theoretical point of view. These problems are more rigorously graded, with a high expectation on the student providing clear, detailed, and justified answers. Lastly, you may work with other students and ask me any questions, but you may not look up solutions online. You must write your solutions independently so I may interpret your understanding while grading. 
\end{abstract}

\subsection*{Book Problems}
\begin{itemize}
\item   [\S 4.1:] 3, 7, 18, 20, 32, 44, 45
\item   [\S 4.2:] 6, 8, 18, 22, 29, 30, 41
\item   [\S 4.3:] 11, 12, 22, 34, 38, 48, 50
\item   [\S 4.4:]	8, 15, 18, 19, 35, 41, 67
\end{itemize}

\subsection*{Other Problems}

\paragraph*{Problem 1.} For simple pendulums, the period $T$ may be described by
\[
T=2\pi\sqrt{\frac{L}{g}},
\]
where $L$ is the length of the pendulum and $g$ is the constant acceleration of gravity. If the pendulum is made of metal, then its length will vary with temperature as follows
\[
\frac{dL}{du}=kL,
\]
where $u$ is the temperature and $k$ is a constant. Find the rate at which the period changes with respect to the temperature. 

\end{document}