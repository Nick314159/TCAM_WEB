\documentclass{article}
\usepackage{titling}

\setlength{\droptitle}{-10em}   % This is your set screw

\makeatletter
\renewcommand{\abstractname}{Instructions}
\makeatother

\title{MAT -- 112: Calculus I and Modeling\\
\large{Homework 8}}
\author{Instructor: Thomas R. Cameron}
\date{Due: 4/13/2018}

\begin{document}
\maketitle

\begin{abstract}
You must complete all book problems and other problems. The book problems are intended to give you practice in solving problems from the textbook. They are graded based upon completion and correctness. The other problems are intended to help further your understanding of the concepts from a theoretical point of view. These problems are more rigorously graded, with a high expectation on the student providing clear, detailed, and justified answers. Lastly, you may work with other students and ask me any questions, but you may not look up solutions online. You must write your solutions independently so I may interpret your understanding while grading. 
\end{abstract}

\subsection*{Book Problems}
\begin{itemize}
\item   [\S 7.1:] 28, 45, 48
\item   [\S 7.2:] 29, 31, 62
\item   [\S 7.3:] 17, 31 (Do 3 estimates, one using left endpoints of the subintervals, one using right endpoints, and one that is the average of the two previous answers), 39 (Find a third estimate by averaging the two answers you get following the instructions in the exercise).
\item	  [\S 7.4:] 7, 14, 31, 63
\end{itemize}

\subsection*{Other Problems}

\paragraph*{Problem 1.} State the Fundamental Theorem of Calculus (on p. 398 of your book) and explain how it connects the concept of the definite and indefinite integral. 

\paragraph*{Problem 2.} Use the Fundamental Theorem of Calculus to justify the curious relationship among differentials: If $y=f(x)$, then 
\[
dy=f^{'}(x)dx.
\]

\paragraph*{Problem 3.} State the second part of the Fundamental Theorem of Calculus (discussed in class on 4/6) and explain how it connects the concept of the definite integral and the derivative. 

\paragraph*{Problem 4.} Use the second part of the Fundamental Theorem of Calculus to show that if $u$ and $v$ are differentiable functions of $x$, then
\[
\frac{d}{dx}\left(\int_{u}^{v}f(t)dt\right)=f(v)\frac{dv}{dx}-f(u)\frac{du}{dx}.
\]

\end{document}