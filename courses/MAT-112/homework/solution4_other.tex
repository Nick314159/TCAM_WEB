\documentclass{article}
\usepackage{amsmath}
\usepackage{amssymb}
\usepackage{enumerate}

\title{MAT -- 112: Calculus I and Modeling\\
\large{Solution 4}}
\author{Thomas R. Cameron}
\date{2/23/2018}

\begin{document}
\maketitle

\subsection*{Other Problems}

\paragraph*{Problem 1.}	We can find the rate at which the period changes with respect to the temperature ($dT/du$) by taking the derivative of the equation
\[
T=2\pi\sqrt{\frac{L}{g}}
\]
with respect to $u$. To that end, we have
\[
\frac{dT}{du}=2\pi\frac{d}{du}\sqrt{\frac{L}{g}},
\]
which requires the chain rule with outside function $\sqrt{x}$ and inside function $\frac{L}{g}$. Thus, we have
\begin{align*}
\frac{dT}{du}&=2\pi\left(\frac{1}{2}\left(\frac{L}{g}\right)^{-1/2}\frac{1}{g}\frac{dL}{du}\right) \\
&=\pi\left(\frac{L}{g}\right)^{-1/2}\frac{1}{g}(kL) \\
&=k\pi\sqrt{\frac{L}{g}}.
\end{align*}

\end{document}