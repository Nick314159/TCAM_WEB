\documentclass{article}
\usepackage{amsmath}
\usepackage{amssymb}
\usepackage{enumerate}

\title{MAT -- 112: Calculus I and Modeling\\
\large{Solution 6}}
\author{Thomas R. Cameron}
\date{3/23/2018}

\begin{document}
\maketitle

\subsection*{Other Problems}

\paragraph*{Problem 1.} The Volume of a right circular cone is given by
\[
V=\frac{\pi}{3}r^{2}h,
\]
where $r$ is the radius of the circle and $h$ is the height of the cone. In our current scenario we have $r=x$ and $h=y+3$. Furthermore, since the cone is inscribed by a sphere of radius $3$ we have $x^{2}+y^{2}=3^{2}$. Therefore, we can write the volume as a function of $y$:
\begin{align*}
V(y)&=\frac{\pi}{3}(9-y^{2})(y+3)\\
&=\frac{\pi}{3}(27+9y-3y^{2}-y^{3}),
\end{align*}
where $0\leq y\leq 3$. Therefore, we can find the absolute max volume of the cone by finding the critical points and comparing the volume at the critical points and end points. Note that
\[
V^{'}(y)=\frac{\pi}{3}(9-6y-3y^{2}),
\]
which is zero when $y=-3$ and $y=1$. Furthermore,
\begin{align*}
V(0)&=9\pi, \\
V(1)&=\frac{32}{3}\pi, \\
V(3)&=0.
\end{align*}
Therefore, the max volume of the cone is $\frac{32}{3}\pi$ and occurs when $y=1$ and $x=\sqrt{8}$. 

\end{document}