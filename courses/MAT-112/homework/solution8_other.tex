\documentclass{article}
\usepackage{amsmath}
\usepackage{amssymb}
\usepackage{enumerate}
\usepackage{siunitx}

\title{MAT -- 112: Calculus I and Modeling\\
\large{Solution 8}}
\author{Thomas R. Cameron}
\date{3/23/2018}

\begin{document}
\maketitle

\subsection*{Other Problems}

\paragraph*{Problem 1.} Let $f(x)$ be continuous on the interval $[a,b]$ and let $F(x)$ be any antiderivative of $f(x)$. Then
\[
\int_{a}^{b}f(x)dx=F(b)-F(a).
\]
Recall that the indefinite integral
\[
\int f(x)dx=F(x)+C
\]
denotes the entire family of antiderivatives of $f(x)$. The Fundamental Theorem of Calculus (part 1) states that you can take any function from that family of antiderivatives and use it to compute the definite integral. 

\paragraph*{Problem 2.} Let $y=f(x)$ and let $a$ and $b$ be any two real numbers such that $f(x)$ is continuous on the interval $[a,b]$. Furthermore, define $y_{1}=f(a)$ and $y_{2}=f(b)$. Then
\begin{align*}
\int_{a}^{b}f^{'}(x)dx&=f(b)-f(a)~\text{(by the F.T.C.)} \\
&=y_{2}-y_{1} \\
&=\int_{y_{1}}^{y_{2}}dy~\text{(by the F.T.C.)}
\end{align*}
It follows that we have further justification for the curious relationship among differentials; that is,
\[
dy = f^{'}(x)dx.
\]

\paragraph*{Problem 3.} Let $f(x)$ be a continuous function on an open interval $[a,b]$ and define
\[
F(x)=\int_{a}^{x}f(t)dt,
\]
then $F^{'}(x)=f(x)$. Recall that we call $F(x)$ an accumulator function and it measures the growth or decay in the area bounded by $f$, the $x$-axis, and the interval $[a,x]$. The Fundamental Theorem of Calculus (part 2) states that the rate of change of the accumulator function is equal to the integrand evaluated at $x$, i.e., $f(x)$. Another interpretation lies in noting that
\[
\frac{d}{dx}\left(\int_{a}^{x}f(t)dt\right)=f(x).
\]
Thus, the derivative and integral are pseudo-inverses. 

\paragraph*{Problem 4.} Let $f(x)$ be a continuous function and let $c$ be any real number. If $u$ and $v$ are differential functions of $x$, then
\begin{align*}
\frac{d}{dx}\left(\int_{u}^{v}f(t)dt\right) &= \frac{d}{dx}\left(\int_{u}^{c}f(t)dt+\int_{c}^{v}f(t)dt\right)~\text{(by property 4 on p. 400)} \\
&= \frac{d}{dx}\left(-\int_{c}^{u}f(t)dt+\int_{c}^{v}f(t)dt\right)~\text{(by property 5 on p. 400)}.
\end{align*}
Define $G(x)=\int_{c}^{x}f(t)dt$, then $G^{'}(x)=f(x)$ by the F.T.C. Furthermore, by the chain rule we have
\[
\frac{d}{dx}\int_{c}^{u}f(t)dt = G^{'}(u)\frac{du}{dx}
\]
and
\[
\frac{d}{dx}\int_{c}^{v}f(t)dt = G^{'}(v)\frac{dv}{dx}.
\]
Therefore,
\[
\frac{d}{dx}\left(\int_{u}^{v}f(t)dt\right) = f(v)\frac{dv}{dx}-f(u)\frac{du}{dx}.
\]


\end{document}