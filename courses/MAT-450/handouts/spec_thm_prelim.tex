\documentclass{article}
\usepackage{amsmath}
\usepackage{amsthm}
\usepackage{amssymb}

\newtheorem{theorem}{Theorem}

\title{Spectral Theorem Preliminaries}
\author{Instructor: Thomas R. Cameron}
\date{}
\begin{document}
\maketitle

Let $V$ be an inner product space over the field $\mathbb{F}$. We will make use of the following results.

\begin{theorem}
Let $W$ be a finite-dimensional subspace of $V$ and let $y\in V$. Then there exists a unique $u\in W$ and $z\in W^{\perp}$ such that $y=u+z$. Furthermore, if $\{v_{1},\ldots,v_{k}\}$ is a basis for $W$, then 
\[
u=\sum_{i=1}^{k}\langle y,v_{i}\rangle v_{i}
\]
\end{theorem}
\begin{proof}
See~\cite[Theorem 6.6]{Friedberg}. 
\end{proof}
This in fact holds for any closed subspace $W$. Hence, we can write $V=W\oplus W^{\perp}$ for any closed subspace $W$ of $V$. 

\begin{theorem}
If $V$ is finite-dimensional and $\mathbb{F}=\mathbb{C}$. Then $T\in\mathcal{L}(V)$ is normal if and only if there exists an orthonormal basis for $V$ consisting of eigenvectors of $T$. 
\end{theorem}
\begin{proof}
See~\cite[Theorem 6.16]{Friedberg}.
\end{proof}

\begin{theorem}
If $V$ is finite-dimensional and $\mathbb{F}=\mathbb{R}$. Then $T\in\mathcal{L}(V)$ is self-adjoint if and only if there exists an orthonormal basis for $V$ consisting of eigenvectors of $T$. 
\end{theorem}
\begin{proof}
See~\cite[Theorem 6.17]{Friedberg}.
\end{proof}

\begin{thebibliography}{1}
\bibitem{Friedberg}
S.H. Friedberg, A.H. Insel, and L.E. Spence.
\textit{Linear Algebra}.
Pearson Education, Upper Saddle River, NJ, 4th edition, 2003.
\end{thebibliography}

\end{document}