\documentclass{article}
\usepackage{amsmath}
\usepackage{amssymb}
\usepackage{enumerate}
\usepackage{hyperref}
\hypersetup{
    colorlinks=true,
    linkcolor=blue,
    filecolor=magenta,      
    urlcolor=cyan,
}

\makeatletter
\renewcommand{\abstractname}{Instructions}
\makeatother

\newcommand{\spn}{{span\mkern1mu}}

\title{MAT -- 450: Advanced Linear Algebra\\
\large{Homework 5}}
\author{Instructor: Thomas R. Cameron}
\date{Due: 4/6/2018}

\begin{document}
\maketitle

\begin{abstract}
There has been some discussion in class regarding the direct sum, and the null space and range of a linear operator. This small homework assignment is intended to clarify our discussion and give you a stronger understanding of the details. 
\end{abstract}

\paragraph*{Problem 1.} Let $V$ be a finite-dimensional inner product space and $T\in\mathcal{L}(V)$.
\renewcommand{\theenumi}{\alph{enumi}}
\begin{enumerate}
\item	Suppose that $T=T^{*}$ and show that $V=R(T)\oplus N(T)$.
\item	Suppose that $R(T)$ and $N(T)$ are non-trivial sets. If $V=R(T)\oplus N(T)$ is it necessary that $T=T^{*}$. If not, then provide a counterexample.
\item	Provide an example where $V=\mathbb{R}^{2}$ is the direct sum of $R(T)$ and $N(T)$, but $N(T)\neq R(T)^{\perp}$. As a connection to part b., note that in this case it is not possible for $T=T^{*}$. 
\item	Prove that there exists a $k\in\mathbb{N}$ such that $V=R(T^{k})\oplus N(T^{k})$. 
\end{enumerate}

\end{document}