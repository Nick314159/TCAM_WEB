\documentclass{exam}
\usepackage[utf8]{inputenc}
\usepackage{amsmath}
\usepackage{amssymb}
\usepackage{graphicx}
\usepackage{multicol}

\newcommand{\class}{MAT -- 450}
\newcommand{\term}{Spring 2018}
\newcommand{\examnum}{Review 2}
\newcommand{\examdate}{Dates: 3/15 -- 3/20}

\pagestyle{head}
\firstpageheader{}{}{}
\runningheader{\textbf{Start Time:}}{\textbf{End Time:}}{}
\runningheadrule

\begin{document}

\noindent
\begin{tabular*}{\textwidth}{l @{\extracolsep{\fill}} r @{\extracolsep{6pt}} l}
\textbf{\class} & \textbf{Name:} & \makebox[2in]{\hrulefill}\\
\textbf{\term} &&\\
\textbf{\examnum} &&\\
\textbf{\examdate} & \textbf{Pledge:}	& \makebox[2in]{\hrulefill}\\
\end{tabular*}\\
\rule[2ex]{\textwidth}{2pt}

\begin{center}
\fbox{\fbox{\parbox{5.5in}{\centering
Each question topic and the point value is recorded in the tables below. You may review these topics from any resource at your leisure. Once you decide to start a review problem, you are on the clock and you must work without any external resources. Each problem can be done one at a time but must be finished in a single sitting. Answer each question in the space provided, if you run out of space, then you may continue on the back of the page. It is your responsibility to plan out your time to ensure that you can finish all problems within the $4.0$ hours allotted. By writing your name and signing the pledge you are stating that your work adheres to these terms and the Davidson honor code. }}}
\end{center}

\vspace*{1em}

\begin{multicols}{2}
%% Scoring Table
\centering
\textbf{Scoring Table}\\
\addpoints
\gradetable[v][questions]

\columnbreak
%% Topics Table
\centering
\textbf{Topics Table}\\
\renewcommand{\arraystretch}{1.5}
\begin{tabular}{| c | c |}
\hline
Question & Topic\\
\hline
1 & Invariant Subspaces \\
\hline
2 & Eigenspaces \\
\hline
3 &  The Determinant \\
\hline
4 & Cayley-Hamilton Theorem \\
\hline
&\\
\hline
\end{tabular}
\end{multicols}

\vspace*{2em}
%%Time Table
\begin{center}
\textbf{Time Table}\\
\renewcommand{\arraystretch}{2.5}
\begin{tabular}{| c | c | c | c | c | c | c | c | c | }
\hline
Question & ~~~~1~~~~ & ~~~~2~~~~ & ~~~~3~~~~ & ~~~~4~~~~ \\
\hline
Time & & & &  \\
\hline
\end{tabular}
\end{center}

\newpage

\begin{questions}

\question
\begin{parts}
\part[2] Let $T\colon~V\rightarrow~W$ be linear, where $V$ and $W$ are vector spaces over field $\mathbb{F}$. Define $T$-invariant subspaces of $V$.
\vspace{\stretch{1}}
\part[2] Show by example that there are linear operators with no non-trivial invariant subspace. 
\vspace{\stretch{1}}
\part[6] Let $T\colon~V\rightarrow~V$ be linear, where $V$ is a vector space over field $\mathbb{F}$. Let $W$ be a $T$-invariant subspace of $V$ and let $p\in\mathcal{P}(\mathbb{F})$. Prove that $W$ is also $p(T)$-invariant.
\vspace{\stretch{2}}
\end{parts}

\newpage

\question Let $T\colon~V\rightarrow~V$ be linear with $V$ a finite dimensional vector space over $\mathbb{C}$. 
\begin{parts}
\part[4] State the definition of an eigenvalue and eigenvector pair of $T$, and state the definition of an eigenspace of $T$.
\vspace{\stretch{1}}
\part[6] Prove, without using determinants, that $T$ has an eigenvalue. 
\vspace{\stretch{3}}
\end{parts}

\newpage

\question Let $T\colon~V\rightarrow~V$ be linear with $V$ a finite dimensional vector space over $\mathbb{R}$. 
\begin{parts}
\part[2] Define the determinant of $T$ in terms of its action, volume magnification, and orientation change.
\vspace{\stretch{1}}
\part[4] Use the definition in (a) to prove that if $T$ is invertible then $\det(T^{-1})=1/\det(T)$.
\vspace{\stretch{1}}
\part[4] The algebraic multiplicity of an eigenvalue is defined as the multiplicity the eigenvalue has as a root of the characteristic polynomial: $\det(\lambda I-T)$. The geometric multiplicity of an eigenvalue is defined as the dimension of the eigenspace: $E(\lambda,T)$. Prove that $T$ is diagonalizable if and only if the algebraic and geometric multiplicity of every eigenvalue is equal. 
\vspace{\stretch{1}}
\end{parts}

\newpage

\question Let $T\colon~V\rightarrow~V$ be linear with $V$ a finite dimensional vector space over field $\mathbb{F}$. 
\begin{parts}
\part[3] State the Cayley-Hamilton theorem. 
\vspace{\stretch{1}}
\part[6] The Cayley-Hamilton theorem relies on two results
\begin{enumerate}
\item	For any $T$-invariant subspace $W$ of $V$, the characteristic polynomial of $T_{W}$ divides the characteristic polynomial of $T$.
\item	Let $x\in V$ be nonzero and define $W=\text{span}(x,Tx,T^{2}x,\ldots)$. If $\dim(W)=k$ and 
\[
a_{0}x+a_{1}Tx+\cdots+a_{k-1}T^{k-1}x+a_{k}T^{k}x=0,
\]
then the characteristic polynomial of $T_{W}$ is $f(t)=a_{0}+a_{1}t+\cdots+a_{k-1}t^{k-1}+t^{k}$. 
\end{enumerate}
Use these two results to provide a proof of the Cayley-Hamilton theorem.
\vspace{\stretch{2}}

\newpage

\part[6] Let
\[
A=\begin{bmatrix}-1 & 0 & 0 & 0 & 0\\0 & 2 & 1 & 0 & 0\\0 & 0 & 2 & 0 & 0\\0 & 0 & 0 & -1 & 1\\0 & 0 & 0 & 0 & -1 \end{bmatrix}
\]
Find the characteristic polynomial of $A$ and show that it satisfies the Cayley-Hamilton theorem. Then find a smaller degree polynomial that also satisfies the Cayley-Hamilton theorem. \emph{Hint: try a polynomial with the same roots, but different multiplicities}. Also, you may use Mathematica (or any other software) to help with the arithmetic. 
\vspace{\stretch{2}}
\end{parts}

\end{questions}

\end{document}